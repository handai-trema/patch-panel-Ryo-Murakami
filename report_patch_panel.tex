\documentclass[10pt,a4paper,onecolumn]{jarticle}
\usepackage[dvips]{graphicx}
\usepackage{subfigure}
\usepackage{amsmath}
%\usepackage{listings}
%\usepackage{jlisting}
\usepackage{ascmac}
\usepackage{here}
\usepackage{txfonts}
\usepackage{listings,jlisting}
\renewcommand{\lstlistingname}{リスト}
\lstset{language=c,
  basicstyle=\ttfamily\scriptsize,
  commentstyle=\textit,
  classoffset=1,
  keywordstyle=\bfseries,
  frame=tRBl,
  framesep=5pt,
  showstringspaces=false,
  numbers=left,
  stepnumber=1,
  numberstyle=\tiny,
  tabsize=2
}
\usepackage[top=25truemm,bottom=25truemm,left=25truemm,right=25truemm]{geometry}
\title{情報ネットワーク学演習 2\\ 課題: パッチパネルの拡張}
\author{33E16022 村上 遼}
\date{\today}

\begin{document}

\maketitle

\section{課題内容}

パッチパネルに機能を追加する.
授業で説明のあったパッチの追加と削除以外に,以下の機能を追加する.

\begin{enumerate}
	\item ポートのミラーリング
	\item パッチとポートミラーリングの一覧
\end{enumerate}


\section{課題に対する回答}


\subsection{ミラーリングの追加}

辻健太さんの回答を元にソースコードの変更を考案した.
ミラーリングの作成のために,create\_mirror メッソドと add\_mirror\_entries メソッドを追加した.
リスト \ref{lst:mirror-add} に create\_mirror メッソドと add\_mirror\_entries メソッドを示す.

\begin{lstlisting}[caption = ミラーリング作成用のメソッド, label = lst:mirror-add]
  def create_mirror(dpid, port_monitor, port_mirror)
		add_mirror_entries dpid, port_monitor, port_mirror
		@mirror[dpid] += [[port_monitor, port_mirror]]
  end

 def add_mirror_entries(dpid, port_monitor, port_mirror)
		send_flow_mod_delete(dpid, match: Match.new(in_port: port_mirror))
		for patch_tmp in @patch[dpid].each do
			port_in = patch_tmp[0]
			port_out = patch_tmp[1]
			if port_in == port_monitor then
				send_flow_mod_delete(dpid, match: Match.new(in_port: port_in))
				send_flow_mod_add(dpid, 
								match: Match.new(in_port: port_monitor),
								actions: [
								SendOutPort.new(port_out),
								SendOutPort.new(port_mirror)
								])
			end		
		end
 end
\end{lstlisting}

create\_mirror メソッドでは,入力された情報を元に変数 @mirror にミラーリングの情報を追加している.
モニタ用のポートとミラーリング用のポートを区別するために,変数 @mirror に追加するときにソートを行っていない.

add\_mirror\_entries では,一度スイッチ側の FlowTable 規則を削除してから,ミラーリング生成に必要な FlowTable の情報追加を行っている.

\subsection{ミラーリングの削除}

ミラーリングの削除のために,delete\_mirror メッソドと delete\_mirror\_entries メソッドを追加した.
リスト \ref{lst:mirror-add} に create\_mirror メッソドと delete\_mirror\_entries メソッドを示す.

\begin{lstlisting}[caption = ミラーリング削除用のメソッド, label = lst:mirror-delete]
def delete_mirror(dpid, port_monitor, port_mirror)
	if @mirror[dpid].include?([port_monitor, port_mirror]) then
		delete_mirror_entry dpid, port_monitor, port_mirror
		@mirror[dpid] -= [[port_monitor, port_mirror]]
	end
end

def delete_mirror_entry(dpid, port_monitor, port_mirror)
	send_flow_mod_delete(dpid, match: Match.new(in_port: port_monitor))
	for patch_tmp in @patch[dpid].each do
		port_in = patch_tmp[0]
		port_out = patch_tmp[1]
		if port_in == port_monitor then
			send_flow_mod_add(dpid,
							match: Match.new(in_port: port_in),
							actions: SendOutPort.new(port_out))
		end
	end
end
\end{lstlisting}

delete\_mirror メソッドでは,変数 @mirror にミラーリングが登録されていた場合,変数 @mirror からミラーリングの情報を削除している.

delete\_mirror\_entry メソッドでは,一度スイッチ側の FlowTable を削除してから,ミラーリング削除後に必要な FlowTable の情報を追加している.

\subsection{出力}

情報の出力のために,dump メソッドを追加した.
リスト \ref{lst:dmp} に dump メソッドを示す.

\begin{lstlisting}[caption = dump, label = lst:dmp]
def dump(dpid)
    	str = "Patches:\n"
    	for patch_tmp in @patch[dpid].each do
      		port_in = patch_tmp[0]
      		port_out = patch_tmp[1]
      		str += "\t"
      		str += port_in.to_s
      		str += "<->"
      		str += port_out.to_s
      		str += "\n"
    	end
    
	str += "Mirrors:\n"
    	for mirror_tmp in @mirror[dpid].each do
      		port_monitor = mirror_tmp[0]
      		port_mirror = mirror_tmp[1]
      		str += "\t"
      		str += port_monitor.to_s
      		str += "->"
      		str += port_mirror.to_s
      		str += "\n"
    	end
	str
end
\end{lstlisting}

dump メソッドでは,ミラーリングの情報だけでなく,接続情報も出力している.
変数 str に情報を格納して,\ref{lst:dmp} の 28 行目で出力している.

\section{動作確認}

動作確認のため,trema run ./lib/patch\_panel.rb -c patch\_panel.conf を実行した.
しかしながら,エラーが発生して実効することができなかった.
リスト \ref{lst:result} にエラーメッセージを示す.

\begin{lstlisting}[caption = 動作結果, label = lst:result]
ensyuu2@ensyuu2-VirtualBox:~/week3/patch-panel-Ryo-Murakami$ trema run ./lib/patch_panel.rb 
	-c patch_panel.conf
/home/ensyuu2/.rvm/rubies/ruby-2.2.5/lib/ruby/2.2.0/drb/unix.rb:41:in `initialize': Address already 
	in use - connect(2) for /tmp/trema.PatchPanel.ctl (Errno::EADDRINUSE)
	from /home/ensyuu2/.rvm/rubies/ruby-2.2.5/lib/ruby/2.2.0/drb/unix.rb:41:in `open'
	from /home/ensyuu2/.rvm/rubies/ruby-2.2.5/lib/ruby/2.2.0/drb/unix.rb:41:in `open_server'
	from /home/ensyuu2/.rvm/rubies/ruby-2.2.5/lib/ruby/2.2.0/drb/drb.rb:767:in `block in open_server'
	from /home/ensyuu2/.rvm/rubies/ruby-2.2.5/lib/ruby/2.2.0/drb/drb.rb:765:in `each'
	from /home/ensyuu2/.rvm/rubies/ruby-2.2.5/lib/ruby/2.2.0/drb/drb.rb:765:in `open_server'
	from /home/ensyuu2/.rvm/rubies/ruby-2.2.5/lib/ruby/2.2.0/drb/drb.rb:773:in `open_server'
	from /home/ensyuu2/.rvm/rubies/ruby-2.2.5/lib/ruby/2.2.0/drb/drb.rb:1405:in `initialize'
	from /home/ensyuu2/.rvm/rubies/ruby-2.2.5/lib/ruby/2.2.0/drb/drb.rb:1695:in `new'
	from /home/ensyuu2/.rvm/rubies/ruby-2.2.5/lib/ruby/2.2.0/drb/drb.rb:1695:in `start_service'
	from /home/ensyuu2/.rvm/gems/ruby-2.2.5/gems/trema-0.9.0/lib/trema/command.rb:124:
		in `start_controller_and_drb_threads'
	from /home/ensyuu2/.rvm/gems/ruby-2.2.5/gems/trema-0.9.0/lib/trema/command.rb:51:in `run'
	from /home/ensyuu2/.rvm/gems/ruby-2.2.5/gems/trema-0.9.0/bin/trema:54:in `block (2 levels) in <module:App>'
	from /home/ensyuu2/.rvm/gems/ruby-2.2.5/gems/gli-2.13.4/lib/gli/command_support.rb:126:in `call'
	from /home/ensyuu2/.rvm/gems/ruby-2.2.5/gems/gli-2.13.4/lib/gli/command_support.rb:126:in `execute'
	from /home/ensyuu2/.rvm/gems/ruby-2.2.5/gems/gli-2.13.4/lib/gli/app_support.rb:296:in `block in call_command'
	from /home/ensyuu2/.rvm/gems/ruby-2.2.5/gems/gli-2.13.4/lib/gli/app_support.rb:309:in `call'
	from /home/ensyuu2/.rvm/gems/ruby-2.2.5/gems/gli-2.13.4/lib/gli/app_support.rb:309:
		in `call_command'
	from /home/ensyuu2/.rvm/gems/ruby-2.2.5/gems/gli-2.13.4/lib/gli/app_support.rb:83:in `run'
	from /home/ensyuu2/.rvm/gems/ruby-2.2.5/gems/trema-0.9.0/bin/trema:268:in `<module:App>'
	from /home/ensyuu2/.rvm/gems/ruby-2.2.5/gems/trema-0.9.0/bin/trema:14:in `<module:Trema>'
	from /home/ensyuu2/.rvm/gems/ruby-2.2.5/gems/trema-0.9.0/bin/trema:12:in `<top (required)>'
	from /home/ensyuu2/.rvm/gems/ruby-2.2.5/bin/trema:23:in `load'
	from /home/ensyuu2/.rvm/gems/ruby-2.2.5/bin/trema:23:in `<main>'
	from /home/ensyuu2/.rvm/gems/ruby-2.2.5/bin/ruby_executable_hooks:15:in `eval'
	from /home/ensyuu2/.rvm/gems/ruby-2.2.5/bin/ruby_executable_hooks:15:in `<main>'
\end{lstlisting}

エラーメッセージの原因を特定することができなかった.
書き換えた ruby ファイルではない場所からエラーが発生しているため,環境構築が上手く出来ていない可能性がある. 

\end{document}
